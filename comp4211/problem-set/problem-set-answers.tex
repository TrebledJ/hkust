%%% Packages %%%
\usepackage{algorithm}		% Algorithms with captions (\begin{algorithm}).
\usepackage{algpseudocode}	% Algorithm pseudocode (\begin{algorithmic}).
\usepackage{amsfonts} 		% For extra mathy stuff (\mathbb).
\usepackage{amsmath} 		% For section headers.
\usepackage{amssymb} 		% For extra mathy stuff, e.g. \therefore.
\usepackage{amsthm}  		% Custom theorem styles.
\usepackage[toc,page]{appendix} % Appendices.
\usepackage{array} 			% Special tabular.
\usepackage{braket} 		% For \set \bra \ket commands.
\usepackage{caption} 		% Caption setup.
\usepackage{centernot} 		% For \centernot.
\usepackage{enumitem} 		% For \setlist.
\usepackage[T1]{fontenc} 	% Code blocks!
\usepackage[margin=1in]{geometry} % global page margin
\usepackage{hyperref} 		% For links.
\usepackage[latin1]{inputenc} % For fancy text input
\usepackage{lipsum} 		% Lorem ipsum <3
\usepackage{listings} 		% Code blocks!
\usepackage{longtable} 		% For the longtable env.
\usepackage{mathtools} 		% \DeclarePairedDelimiter
\usepackage{multirow}		%
\usepackage{scrextend} 		% For the ddmargin env.
\usepackage[inference]{semantic} % Logical inference.
\usepackage{setspace} 		% Custom line spacing.
\usepackage{textcomp} 		% Code blocks.
\usepackage{url} 			% URLs in bibliography.
\usepackage{verbatim} 		% Multiline comments.
\usepackage{wrapfig} 		% Text-wrapping around figures.
\usepackage[svgnames]{xcolor} % For some pretty colours; see https://www.latextemplates.com/svgnames-colors.
\usepackage{ulem}  			% For \sout.

\usepackage{graphicx} 		% Graphics~
\graphicspath{{img/}}

\usepackage{booktabs, makecell, tabularx}

% \usepackage{apacite}
\usepackage[sort,comma,nonamebreak]{natbib}
% \bibliographystyle{apalike}
\bibliographystyle{plainnat}
% \bibliographystyle{apacite}


\usepackage{tikz} % Tikz!!!
\usepackage{tikz-qtree} % Simple trees!
\usetikzlibrary{arrows}

\usepackage{fancyhdr} % Um... Fancy headers!
\pagestyle{fancy}

\usepackage{mathptmx} % Times New Romans-esque font.

%%% Tex Commands. %%%

% Side-by-side figures with captions!
% \DeclareCaptionLabelFormat{cont}{#1~#2\alph{ContinuedFloat}}
% \captionsetup[ContinuedFloat]{labelformat=cont}

% Convenience wrapper for bracketing. \delim* for auto-sizing, \delim[\bigg] for manual sizing.
\DeclarePairedDelimiter\ceil{\lceil}{\rceil}
\DeclarePairedDelimiter\floor{\lfloor}{\rfloor}
\DeclarePairedDelimiter\abs{\lvert}{\rvert}
\DeclarePairedDelimiter\parens{(}{)}
\DeclarePairedDelimiter\angles{\langle}{\rangle}

%%%% Useful Macros %%%
\renewcommand{\o}{\varnothing} % Empty set.
\newcommand{\N}{\mathbb{N}} % Natural numbers.
\newcommand{\Z}{\mathbb{Z}} % Integers.
\newcommand{\Q}{\mathbb{Q}} % Rational numbers.
\newcommand{\R}{\mathbb{R}} % Real numbers.
\newcommand{\C}{\mathbb{C}} % Complex numbers.
\newcommand{\powerset}{\mathcal{P}} % Power set.
\newcommand{\insum}{\textstyle\sum} % Inline summation.
\newcommand{\st}{\text{ such that }} % Inline summation.

\renewcommand{\Return}{\State \textbf{return }} % Pseudocode return statement. To be used with the algorithmic package.

\renewcommand{\epsilon}{\varepsilon}
\renewcommand{\b}[1]{\!\left(#1\right)}

\newcommand{\xor}{\oplus}
\newcommand{\cnot}{\centernot}

\DeclareRobustCommand{\abinom}{\genfrac{\langle}{\rangle}{0pt}{}}

\newcommand{\code}[1]{\texttt{#1}}

% \includecode[<caption>]{filename}
\newcommand{\includecode}[2][]{\lstinputlisting[caption=#1, escapechar=, style=custom_python]{#2}}

% Usage:
% \begin{graph}
% 	\node{parent}
%		child {node {child 1}}
% 		child {node {child 2}};
% \end{graph}
\newenvironment{graph}[1]
	{
		\begin{minipage}{#1}
			\centering
			\begin{tikzpicture}
				\tikzset{filled/.style = {shape=circle,fill,minimum size=0.1em}}
				\tikzset{edge/.style = {-,> = latex'}}
				\tikzset{arrow/.style = {->,> = latex'}}
				\tikzset{sibling distance=6em, every node/.style = {shape=circle,draw,minimum size=1em}}
	}
	{
			\end{tikzpicture}
		\end{minipage}
	}

% \newenvironment{tree}[1]
% 	{
% 		\begin{graph}[
% 			sibling distance=10em,
% 			every tree/.style = {shape=rectangle, rounded corners,
% 			  draw, align=center,
% 			  top color=white, bottom color=blue!20}]]
% 			]{#1}
% 	}
% 	{
% 		\end{graph}
% 	}

% Theorems!!!
\newtheorem{theorem}{Theorem}[section]
\newtheorem*{definition*}{Definition}

% Named theorems!
\makeatletter
\newtheoremstyle{namedStyle}
{\topsep}{\topsep}
{}{}
{\bfseries}{.}
{0.5em}
{\thmname{\@ifempty{#3}{#1}\@ifnotempty{#3}{#3}}}
\makeatother
\theoremstyle{namedStyle}
\newtheorem*{solution}{Solution}
\newtheorem*{namedProof}{Proof}

%%% Document Settings %%% 
% Global line spacing.
% \onehalfspacing % 1.5
\doublespacing % 2.0
% \setstretch{1.25} % Custom.

\setlength\parindent{24pt} % Paragraph indent.
\setlength{\headheight}{15pt}

\renewcommand\tabcolsep{5pt} % Table column spacing.
\renewcommand\arraystretch{0.9} % Matrix spacing (or table row spacing).

%%% List Settings %%%
% https://tex.stackexchange.com/questions/300340/topsep-itemsep-partopsep-parsep-what-do-they-each-mean-and-what-about

% \setlist{nosep} % Set tight ordered-lists.

% Default enumerate.
\setenumerate{
	label=(\alph*),
	topsep=1pt,
	itemsep=1pt,
	parsep=1em, % Separation between paragraphs within items.
	}

\makeatletter
% \skipitems{n: int}: skip items in an enumerated list
\newcommand{\skipitems}[1]{%
	\addtocounter{\@enumctr}{#1}%
}

% Same-line enumerated items. Use \inlineitem instead of \item.
% https://tex.stackexchange.com/a/51089/179128
\newcommand{\inlineitem}[1][]{%
\ifnum\enit@type=\tw@
	{\descriptionlabel{#1}}
	\hspace{\labelsep}%
\else
	\ifnum\enit@type=\z@
		\refstepcounter{\@listctr}\fi
	\quad\@itemlabel\hspace{\labelsep}%
\fi}
\makeatother

	
%%% Misc. Settings %%%

% Have table and figure use same counter.
\makeatletter 
\let\c@table\c@figure
% \let\c@lstlisting\c@figure
\makeatother

% Graphics.
\graphicspath{ {../res/} }

% Set code format.

%% Usage:
% \begin{lstlisting}[style=custom_python(, options=...)]
% # Code goes here.
% \end{lstlisting}

\lstdefinestyle{custom_python}{
	backgroundcolor = \color{WhiteSmoke},
	basicstyle = \ttfamily\small,
	breaklines = true,
	commentstyle = \itshape\color{DarkGreen},   % Comment style.
	deletekeywords = {set},
	escapeinside = {\%*}{*)},                   % For adding LaTeX in code.
	extendedchars = true,
	frame = single,
	keepspaces = true,
	keywordstyle = \bfseries\color{blue},
	language = Python,                   		% Change the programming language here!
	morekeywords = {*, ValueError, as},
	numbers = left, 							% Line-numbers (possible values: none, left, right).
	numbersep = 10pt,                   		% Distance between line-numbers and code
	numberstyle=\small\color{DarkGray}, 		% Style used for line-numbers.
	rulecolor = \color{black},
	showstringspaces = false,
	stringstyle = \color{red},
	tabsize = 4,
	upquote = true
}

\renewcommand{\vec}{\mathbf}

\renewcommand\arraystretch{0.8}

\begin{document}
\lhead{\textit{redacted}}
\chead{COMP4211 Problem Set}
\rhead{\textit{redacted}}

\section{Linear Regression}
\begin{enumerate}[label=(\alph*)]
	\item
	      For the $d = 1$ case, we take the derivative of the loss function $L$ w.r.t. $w_0$ and $w_1$ and set them to 0 to get
	      the following two equations:
	      \begin{align*}
		      \sum_k^N y^{(k)}           & = Nw_0 + w_1 \sum_k^N x_1^{(k)}                                     \\
		      \sum_k^N x_1^{(k)} y^{(k)} & = w_0 \sum_k^N x_1^{(k)} + w_1 \sum_k^N \left( x_1^{(k)} \right)^2.
	      \end{align*}
	      In general for $d > 1$, the second equation is similar for the other derivatives.
	      \begin{align*}
		      \sum_k^N y^{(k)}           & = Nw_0 + \sum_{j=1}^d w_j \sum_k^N x_j^{(k)}                                                         \\
		      \sum_k^N x_i^{(k)} y^{(k)} & = w_0 \sum_k^N x_i^{(k)} + \sum_{j=1}^d w_j \sum_k^N x_j^{(k)} x_i^{(k)} & \text{for } 1 \le i \le d
	      \end{align*}
	      To solve for $\bm{w}$, we can write the above equations in matrix form and factor out $\bm{w}$.
	      \begin{align*}
		      \begin{bmatrix}
			      1 & 1 & \cdots & 1
		      \end{bmatrix}
		      \begin{bmatrix}
			      y^{(1)} \\ y^{(2)} \\ \vdots \\ y^{(N)}
		      \end{bmatrix}
		       & =
		      \begin{bmatrix}
			      N & \sum_k^N x_1^{(k)} & \cdots & \sum_k^N x_d^{(k)}
		      \end{bmatrix}
		      \begin{bmatrix}
			      w_0 \\ w_1 \\ \vdots \\ w_d
		      \end{bmatrix}
		      \\
		      \begin{bmatrix}
			      x_i^{(1)} & x_i^{(2)} & \cdots & x_i^{(N)}
		      \end{bmatrix}
		      \begin{bmatrix}
			      y^{(1)} \\ y^{(2)} \\ \vdots \\ y^{(N)}
		      \end{bmatrix}
		       & =
		      \begin{bmatrix}
			      \sum_k^N x_i^{(k)} & \sum_k^N x_1^{(k)} x_i^{(k)} & \cdots & \sum_k^N x_d^{(k)} x_i^{(k)}
		      \end{bmatrix}
		      \begin{bmatrix}
			      w_0 \\ w_1 \\ \vdots \\ w_d
		      \end{bmatrix}
		       & 1 \le i \le d
		      \\
		      \therefore
		      \begin{bmatrix}
			      1         & 1         & \cdots & 1         \\
			      x_1^{(1)} & x_1^{(2)} & \cdots & x_1^{(N)} \\
			      \vdots    & \vdots    & \ddots & \vdots    \\
			      x_d^{(1)} & x_d^{(2)} & \cdots & x_d^{(N)} \\
		      \end{bmatrix}
		      \begin{bmatrix}
			      y^{(1)} \\ y^{(2)} \\ \vdots \\ y^{(N)}
		      \end{bmatrix}
		       & =
		      \begin{bmatrix}
			      N                  & \sum_k^N x_1^{(k)}           & \cdots & \sum_k^N x_d^{(k)}           \\
			      \sum_k^N x_1^{(k)} & \sum_k^N x_1^{(k)} x_1^{(k)} & \cdots & \sum_k^N x_d^{(k)} x_1^{(k)} \\
			      \vdots             & \vdots                       & \ddots & \vdots                       \\
			      \sum_k^N x_d^{(k)} & \sum_k^N x_1^{(k)} x_d^{(k)} & \cdots & \sum_k^N x_d^{(k)} x_d^{(k)} \\
		      \end{bmatrix}
		      \begin{bmatrix}
			      w_0 \\ w_1 \\ \vdots \\ w_d
		      \end{bmatrix}
	      \end{align*}
	      This can be succinctly expressed as
	      \[
		      \bm{X}^T \bm{y} = \bm{X}^T \bm{X} \bm{w}
	      \]
	      where
	      \[
		      \bm{X} = \begin{bmatrix}
			      1      & x_1^{(1)} & \cdots & x_d^{(1)} \\
			      1      & x_1^{(2)} & \cdots & x_d^{(2)} \\
			      \vdots & \vdots    & \ddots & \vdots    \\
			      1      & x_1^{(N)} & \cdots & x_d^{(N)} \\
		      \end{bmatrix}
		      ,
		      \qquad
		      \bm{y} = \begin{bmatrix}
			      y^{(1)} \\ y^{(2)} \\ \vdots \\ y^{(N)}
		      \end{bmatrix}
		      ,
		      \qquad \text{ and }
		      \bm{w} = \begin{bmatrix}
			      w_0 \\ w_1 \\ \vdots \\ w_d
		      \end{bmatrix}.
	      \]
	      Assuming $\bm{X}^T \bm{X}$ is invertible, we get the
	      least squares estimate:
	      \[
		      \bm{w} = (\bm{X}^T \bm{X})^{-1} \bm{X}^T \bm{y}.
	      \]

	\item The weight update rule can be obtained by taking the negative gradient
	      of the loss function. From slide 9, the gradient of $L(\bm{w}, \bm{S})$ is
	      \[
		      \frac{\partial L}{\partial \bm{w}} = 2\bm{X}^T \bm{X} \bm{w} - 2\bm{X}^T \bm{y}
	      \]
	      Thus, we have the gradient update rule
	      \begin{align*}
		      \bm{w} & \leftarrow \bm{w} - \eta\left(\bm{X}^T \bm{X} \bm{w} - \bm{X}^T \bm{y}\right)
	      \end{align*}
	      where $\eta$ is a tunable learning rate. (The 2 doesn't matter since it's a constant factor.)
\end{enumerate}

\newpage
\section{Logistic Regression}

\begin{enumerate}[label=(\alph*)]
	\item With binary classification, we use the sigmoid $\sigma$ activation function:
	      \[\sigma(z) = \frac{1}{1 + e^{-z}}\]
	      with $z = \bm{w}^T \bm{\hat x}, \ \bm{\hat x} = (x_0, x_1, \dots x_d)$.

	      With multiclass classification, we have a set of weights \textit{for each class}.
	      This means we can apply sigmoid with $z = \bm{w}_i^T \bm{\hat x}$ where each
	      $\bm{w}_i$ are the weights for class $C_i$.

	      Applying sigmoid for each class of weights, we get
	      \[ \sigma(\bm{W} \bm{\hat x}) = \bm{r} \in (0, 1)^K \]
	      where $K$ is the number of classes and $\bm{W}$ are the weights,
	      with the $k$-th row being the weights of class $C_i$.

	      Intuitively, we want the activation output of $\bm{w}_i \bm{\hat x}$ to be close to
	      the actual class $C_i$. That is, $\sigma(\bm{w}_i^T \bm{\hat x}^{(\ell)}) = y_i^{(\ell)}$.

	      %   Let $f(\bm{W}; \bm{x}^{(\ell)}) = \sigma(\bm{W} \bm{\hat x}^{(\ell)})$.
	      Taking a hint from the likelihood function for binary classification, we derive a similar likelihood.
	      %   \begin{align*}
	      %       \prod_{\ell}^N \prod_{i}^K f(\bm{W}; \bm{x}^{(\ell)})^{y_i^{(\ell)}} (1 - f(\bm{W}; \bm{x}^{(\ell)}))^{1 - y_i^{(\ell)}}
	      %   \end{align*}
	      \begin{align*}
		      \prod_{\ell}^N \prod_{i}^K f(\bm{w}_i; \bm{x}^{(\ell)})^{y_i^{(\ell)}} (1 - f(\bm{w}_i; \bm{x}^{(\ell)}))^{1 - y_i^{(\ell)}}
	      \end{align*}
	      with the log likelihood (our loss function) being
	      \[
		      L(\bm{W}, \bm{x}) = -\sum_{\ell}^N \sum_{i}^K y_i^{(\ell)} \log f(\bm{w}_i; \bm{x}^{(\ell)}) + (1 - y_i^{(\ell)}) \log (1 - f(\bm{w}_i; \bm{x}^{(\ell)}))
	      \]

	\item With the formulation from slide 17, the sum of the $K$ outputs equal 1. The formulation considers
	      the total ratio of output and is biased towards dominant values (since we take the exponent $e^{z_j}$).
	      With the formulation presented above, \textit{each} output is in $(0, 1)$, so the sum is not fixed.

	\item With multiclass classification, since we want to find one and only label which
	      the data belongs to, it's more appropriate to use softmax, since usually one class will dominate the loss.

	      The alternative formulation presented here may be more suitable for multilabel classification,
	      which aims to identify multiple labels (i.e. not limited to one) applicable to a sample.
		  For example, we can set some threshold value so that the output classes surpassing that threshold
		  threshold are used as labels.
\end{enumerate}


\newpage
\section{Feedforward Neural Network}

\begin{enumerate}
	\item
	      \begin{align*}
		      \sigma(x) & = \frac{1}{1 + e^{-x}}                                                                            \\
		      \tanh(x)  & = \frac{e^{x} - e^{-x}}{e^x + e^{-x}}                                                             \\
		                & = \frac{e^x + e^{-x} - 2e^{-x}}{e^x + e^{-x}} & \text{add an imaginary } e^{-x} \text{ term}      \\
		                & = 1 - \frac{2e^{-x}}{e^x + e^{-x}}                                                                \\
		                & = 1 - \frac{2}{e^{2x} + 1}                    & \text{multiply numerator and denominator by } e^x \\
		                & = 1 - 2\sigma(-2x)
	      \end{align*}

	\item
	      \begin{align*}
		      \sigma(x)  & = \frac{1}{1 + e^{-x}}                            \\
		      \sigma'(x) & = \frac{e^{-x}}{(1 + e^{-x})^2}                   \\
		                 & = \sigma(x) \frac{e^{-x}}{1 + e^{-x}}             \\
		                 & = \sigma(x) \frac{1 + e^{-x} - 1}{1 + e^{-x}}     \\
		                 & = \sigma(x) \left(1 - \frac{1}{1 + e^{-x}}\right) \\
		                 & = \sigma(x) (1 - \sigma(x))                       \\
		      \tanh(x)   & = 1 - 2\sigma(-2x)                                \\
		                 & = 4\sigma'(-2x)                                   \\
		                 & = 4\sigma(-2x)(1 - \sigma(-2x))
	      \end{align*}

	\item
	      The weight update rules for three layers use the previous weights with
	      \begin{align*}
		      \Delta w^{[2]} & = -\eta \sum_q z^{[1](q)} \sum_\ell \cdots w^{[3]} \cdots \\
		      \Delta w^{[1]} & = -\eta \sum_q x^{(q)} \sum_\ell \cdots w^{[2]} \cdots
	      \end{align*}
	      where $\cdots$ represent a bunch of other factors.

	      The issue is that the weights are all multiplied onto each other, meaning the weights will blow up and changes will likely be drastic.
	      Additionally, the weights in each layer will fluctuate immensely as we apply these weights.

	      Since we're considering units with $\tanh$, another downside is that the large values will saturate, so it becomes hard to differentiate between two nodes.
	      This means we don't get too much meaningful value out of activation and it's difficult to differentiate between different nodes.

	\item
	      In the forward pass, we compute $z = g(\sum wx)$ (a lot of details are stripped, but the general idea remains the same).
	      Since we're supposing $g = \tanh$ and have $w = 0$, so $z = 0$ as well.

	      In the weight update rules, we have $\Delta w = \eta \sum \delta z$.
	      Since all $z = 0$, so all $\Delta w = 0$ as well, and so the weights don't change at all,
	      and the model doesn't learn.

\end{enumerate}


\newpage
\section{Convolutional Neural Networks}

\begin{enumerate}[label=(\alph*)]
	\item
	      \begin{enumerate}[label=(\roman*)]
		      \item The feature map would have size $\floor{\frac{227 + 2p - w}{s} + 1} = \floor{\frac{227 + 4 - 7}{2} + 1} = 113$.
		      \item $(128 \times 7 \times 7 + 1) \times 64 = 401472$ parameters.
	      \end{enumerate}
	\item
	      \begin{enumerate}[label=(\roman*)]
		      \item Size $\floor{\frac{227 + 0 - 1}{1} + 1} = 227$.
		      \item $(128 \times 1 \times 1 + 1) \times 16 = 2064$ parameters.
		      \item Unchanged, with size $\floor{\frac{227 + 4 - 7}{2} + 1} = 113$.
		      \item $(16 \times 7 \times 7 + 1) \times 64 = 50240$ parameters.
		      \item $1 - \frac{50240 + 2064}{401472} \approx 86.97\%$ reduction, $13.03\%$ kept.
	      \end{enumerate}
\end{enumerate}

\newpage
\section{Principal Component Analysis}
\begin{enumerate}[label=(\alph*)]
	\item
	      \begin{align*}
		      \bm{\mu} & = \begin{bmatrix}
			                   \frac{2 + 3 + 4 + 5 + 6 + 7}{6} \\
			                   \frac{1 + 5 + 3 + 6 + 7 + 8}{6}
		                   \end{bmatrix}
		      \\
		               & = \begin{bmatrix}
			                   4.5 \\ 5
		                   \end{bmatrix}
	      \end{align*}

	\item
	      \begin{align*}
		      \mathcal{S} - \bm{\mu} & = \left\{
		      \begin{bmatrix} -2.5 \\ -4 \end{bmatrix},\
		      \begin{bmatrix} -1.5 \\ 0 \end{bmatrix},\
		      \begin{bmatrix} -0.5 \\ -2 \end{bmatrix},\
		      \begin{bmatrix} 0.5 \\ 1 \end{bmatrix},\
		      \begin{bmatrix} 1.5 \\ 2 \end{bmatrix},\
		      \begin{bmatrix} 2.5 \\ 3 \end{bmatrix}
		      \right\}
	      \end{align*}

	\item
	      \[
		      \bm{\Sigma} = \begin{bmatrix}
			      3.5 & 4.4 \\
			      4.4 & 6.8
		      \end{bmatrix}
	      \]

	\item
	      \[ \bm{W} = \begin{bmatrix}
			      0.450798 & 9.849203
		      \end{bmatrix}
	      \]

	      \begin{lstlisting}[style=custom_python,gobble=12]
			import numpy as np
			from math import *

			S = np.array([(2, 1), (3, 5), (4, 3), (5, 6), (6, 7), (7, 8)])
			mu = S.mean(axis=0)

			print('covariance:')
			cov = np.cov(S.T)
			print(cov)

			print('eigenvalues:')
			w, _ = np.linalg.eig(cov)
			print(w)
		  \end{lstlisting}

	\item After taking the first principal component and projecting
	      $\mathcal{S}$ to one dimension, our proportion of variance is
	      \[
		      \frac{9.849203}{9.849203 + 0.450798}
		      \approx 95.6\%
	      \]

\end{enumerate}

\newpage
\section{Clustering -- Partitional Clustering}

\begin{enumerate}[label=(\alph*)]
	\item
	      \begin{itemize}
		      \item $p_1$: \textbf{distance to (4, 2): 5.000}; {distance to (11, 3): 11.180340}
		      \item $p_2$: \textbf{distance to (4, 2): 3.162}; {distance to (11, 3): 10.000000}
		      \item $p_3$: \textbf{distance to (4, 2): 2.828}; {distance to (11, 3): 9.055385}
		      \item $p_4$: \textbf{distance to (4, 2): 2.000}; {distance to (11, 3): 5.099020}
		      \item $p_5$: \textbf{distance to (4, 2): 3.606}; {distance to (11, 3): 5.000000}
		      \item $p_6$: {distance to (4, 2): 4.123}; \textbf{distance to (11, 3): 3.000000}
		      \item $p_7$: {distance to (4, 2): 5.099}; \textbf{distance to (11, 3): 2.828427}
	      \end{itemize}

	\item
	      $C_1$ points: $\{(0, 5), (1, 3), (2, 4), (6, 2), (7, 0)\}$.\\
	      $C_1$ median: $(2, 4)$. \\
	      $C_2$ points: $\{(8, 3), (9, 1)\}$.\\
	      $C_2$ median: $(8.5, 2)$.

	\item
	      \begin{itemize}
		      \item $p_1$ \textbf{distance to (2, 4): 2.236}; {distance to (8.5, 2): 9.013878}
		      \item $p_2$ \textbf{distance to (2, 4): 1.414}; {distance to (8.5, 2): 7.566373}
		      \item $p_3$ \textbf{distance to (2, 4): 0.000}; {distance to (8.5, 2): 6.800735}
		      \item $p_4$ {distance to (2, 4): 4.472}; \textbf{distance to (8.5, 2): 2.500000}
		      \item $p_5$ {distance to (2, 4): 6.403}; \textbf{distance to (8.5, 2): 2.500000}
		      \item $p_6$ {distance to (2, 4): 6.083}; \textbf{distance to (8.5, 2): 1.118034}
		      \item $p_7$ {distance to (2, 4): 7.616}; \textbf{distance to (8.5, 2): 1.118034}
	      \end{itemize}

	\item
	      $C_0$ points: \{(0, 5), (1, 3), (2, 4)\} \\
	      $C_0$ median: (1, 3) \\
	      $C_1$ points: \{(6, 2), (7, 0), (8, 3), (9, 1)\} \\
	      $C_1$ median: (7.5, 1.5)

	\item (1, 3) and (7.5, 1.5).

\end{enumerate}

\newpage
\section{Clustering -- Hierarchical Clustering}

\begin{enumerate}[label=(\alph*)]
	\item $5 - 1 = 4$ merging steps.
	\item $C_1 \leftarrow \{(8, 8), (9, 8)\}$, with distance 1.
	\item $C_2 \leftarrow \{(1, 0), (2, 1)\}$, with distance 1.414.
	\item $C_3 \leftarrow \{(9, 6), \dots C_1\}$, with distance 2.
	\item $C_4 \leftarrow \{\dots C_2, \dots C_3\}$, with distance 8.6023.
	\item For this particular dataset, 2 clusters would be appropriate. The distance between points within clusters $C_2$ and $C_3$ are usually small (1-2), but
	      the distance between the two clusters is large (8.6).
\end{enumerate}


\end{document}