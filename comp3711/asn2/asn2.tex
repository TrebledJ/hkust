\documentclass[12pt,a4paper]{article}
\documentclass[12pt,a4paper]{article}

%%% Packages %%%
\usepackage{algorithm}		% Algorithms with captions (\begin{algorithm}).
\usepackage{algpseudocode}	% Algorithm pseudocode (\begin{algorithmic}).
\usepackage{amsfonts} 		% For extra mathy stuff (\mathbb).
\usepackage{amsmath} 		% For section headers.
\usepackage{amssymb} 		% For extra mathy stuff, e.g. \therefore.
\usepackage{amsthm}  		% Custom theorem styles.
\usepackage[toc,page]{appendix} % Appendices.
\usepackage{array} 			% Special tabular.
\usepackage{bm}             % Bold, italicised math symbols.
\usepackage{braket} 		% For \set \bra \ket commands.
\usepackage{caption} 		% Caption setup.
\usepackage{centernot} 		% For \centernot.
\usepackage{enumitem} 		% For \setlist.
\usepackage[T1]{fontenc} 	% Code blocks!
\usepackage[margin=1in]{geometry} % global page margin
\usepackage{hyperref} 		% For links.
\usepackage[latin1]{inputenc} % For fancy text input
\usepackage{lipsum} 		% Lorem ipsum <3
\usepackage{listings} 		% Code blocks!
\usepackage{longtable} 		% For the longtable env.
\usepackage{mathtools} 		% \DeclarePairedDelimiter
\usepackage{multirow}		%
\usepackage{scrextend} 		% For the ddmargin env.
\usepackage[inference]{semantic} % Logical inference.
\usepackage{setspace} 		% Custom line spacing.
\usepackage{textcomp} 		% Code blocks.
\usepackage{url} 			% URLs in bibliography.
\usepackage{verbatim} 		% Multiline comments.
\usepackage{wrapfig} 		% Text-wrapping around figures.
\usepackage[svgnames]{xcolor} % For some pretty colours; see https://www.latextemplates.com/svgnames-colors.
\usepackage{ulem}  			% For \sout.

\usepackage{float} % Use [H] instead of [h] to force floats. Works with floats in enumerate/itemize environments.
\usepackage[section]{placeins}  % For floats (figures/tables/algorithms) to be sectioned properly.
% Use \FloatBarrier to mark places where a float shouldn't appear beyond.

\usepackage{graphicx} 		% Graphics~
\graphicspath{{../res/img/}}

\usepackage{booktabs, makecell, tabularx}

% \usepackage{apacite}
\usepackage[sort,comma,nonamebreak]{natbib}
% \bibliographystyle{apalike}
\bibliographystyle{plainnat}
% \bibliographystyle{apacite}


\usepackage{tikz} % Tikz!!!
\usetikzlibrary{arrows}

\usepackage{fancyhdr} % Um... Fancy headers!
\pagestyle{fancy}

% \usepackage{mathptmx} % Times New Romans-esque font.

%%% Tex Commands. %%%

% Side-by-side figures with captions!
% \DeclareCaptionLabelFormat{cont}{#1~#2\alph{ContinuedFloat}}
% \captionsetup[ContinuedFloat]{labelformat=cont}

% Convenience wrapper for bracketing. \delim* for auto-sizing, \delim[\bigg] for manual sizing.
\DeclarePairedDelimiter\ceil{\lceil}{\rceil}
\DeclarePairedDelimiter\floor{\lfloor}{\rfloor}
\DeclarePairedDelimiter\abs{\lvert}{\rvert}
\DeclarePairedDelimiter\parens{(}{)}
\DeclarePairedDelimiter\angles{\langle}{\rangle}

%%%% Useful Macros %%%
\renewcommand{\o}{\varnothing} % Empty set.
\newcommand{\N}{\mathbb{N}} % Natural numbers.
\newcommand{\Z}{\mathbb{Z}} % Integers.
\newcommand{\Q}{\mathbb{Q}} % Rational numbers.
\newcommand{\R}{\mathbb{R}} % Real numbers.
\newcommand{\C}{\mathbb{C}} % Complex numbers.
\newcommand{\powerset}{\mathcal{P}} % Power set.
\newcommand{\insum}{\textstyle\sum} % Inline summation.
\newcommand{\st}{\text{ such that }} % Inline text.
\newcommand{\order}{\text{order}} % Inline summation.
\newcommand{\lcm}{\text{lcm}} % Inline summation.
\newcommand{\Ker}[1]{\text{Ker}\parens*{#1}} % Kernel.

\renewcommand{\epsilon}{\varepsilon}
\renewcommand{\b}[1]{\!\left(#1\right)}

\newcommand{\xor}{\oplus}
\newcommand{\cnot}{\centernot}

\DeclareRobustCommand{\abinom}{\genfrac{\langle}{\rangle}{0pt}{}}

\newcommand{\code}[1]{\texttt{#1}}

\newcommand{\deletecommand}[1]{\let#1\undefined}

% \includecode[<caption>]{filename}
\newcommand{\includecode}[2][]{\lstinputlisting[caption=#1, escapechar=, style=custom_python]{#2}}

\newenvironment{graph}[1]
    {
        \begin{minipage}{#1}
            \centering
            \begin{tikzpicture}
                \tikzset{filled/.style = {shape=circle,fill,minimum size=0.1em}}
                \tikzset{edge/.style = {-,> = latex'}}
                \tikzset{arrow/.style = {->,> = latex'}}
                \tikzset{sibling distance=6em, every node/.style = {shape=circle,draw,minimum size=1em}}
    }
    {
            \end{tikzpicture}
        \end{minipage}
    }

% \newenvironment{tree}[1]
% 	{
% 		\begin{graph}[
% 			sibling distance=10em,
% 			every tree/.style = {shape=rectangle, rounded corners,
% 			  draw, align=center,
% 			  top color=white, bottom color=blue!20}]]
% 			]{#1}
% 	}
% 	{
% 		\end{graph}
% 	}

% Theorems!!!
\newtheorem{theorem}{Theorem}[section]
\newtheorem*{definition*}{Definition}

% Named theorems!
\makeatletter
\newtheoremstyle{namedStyle}
{\topsep}{\topsep}
{}{}
{\bfseries}{.}
{0.5em}
{\thmname{\@ifempty{#3}{#1}\@ifnotempty{#3}{#3}}}
\makeatother
\theoremstyle{namedStyle}
\newtheorem*{solution}{Solution}
\newtheorem*{namedProof}{Proof}

%%% Document Settings %%% 
% Global line spacing.
% \onehalfspacing % 1.5
\doublespacing % 2.0
% \setstretch{1.25} % Custom.

\setlength\parindent{24pt} % Paragraph indent.
\setlength{\headheight}{15pt}

\renewcommand\tabcolsep{5pt} % Table column spacing.
\renewcommand\arraystretch{0.9} % Matrix spacing (or table row spacing).

% Break {align} across pages.
\allowdisplaybreaks


%%% List Settings %%%
% https://tex.stackexchange.com/questions/300340/topsep-itemsep-partopsep-parsep-what-do-they-each-mean-and-what-about

% \setlist{nosep} % Set tight ordered-lists.

% Default enumerate.
\setenumerate{
    label=(\alph*),
    topsep=1pt,
    itemsep=1pt,
    parsep=2pt,
    listparindent=\parindent, % Indents within enumerate.
    }

% \skipitems{n: int}: skip items in an enumerated list
\makeatletter
\newcommand{\skipitems}[1]{%
    \addtocounter{\@enumctr}{#1}%
}
\makeatother

    
%%% Misc. Settings %%%

% Have table and figure use same counter.
\makeatletter 
\let\c@table\c@figure
% \let\c@lstlisting\c@figure
\makeatother

% Graphics.
\graphicspath{ {../res/} }

% Set code format.

%% Usage:
% \begin{lstlisting}[style=custom_python(, options=...)]
% # Code goes here.
% \end{lstlisting}

\lstdefinestyle{custom_python}{
    backgroundcolor = \color{WhiteSmoke},
    basicstyle = \ttfamily\small,
    breaklines = true,
    commentstyle = \itshape\color{DarkGreen},   % Comment style.
    deletekeywords = {set},
    escapeinside = {\%*}{*)},                   % For adding LaTeX in code.
    extendedchars = true,
    frame = single,
    keepspaces = true,
    keywordstyle = \bfseries\color{blue},
    language = Python,                   		% Change the programming language here!
    morekeywords = {*, ValueError, as},
    numbers = left, 							% Line-numbers (possible values: none, left, right).
    numbersep = 10pt,                   		% Distance between line-numbers and code
    numberstyle=\small\color{DarkGray}, 		% Style used for line-numbers.
    rulecolor = \color{black},
    showstringspaces = false,
    stringstyle = \color{red},
    tabsize = 4,
    upquote = true
}

\newcommand{\name}{\textit{redacted}}
\newcommand{\sid}{\textit{redacted}}
\newcommand{\thistitle}{COMP3711 Assignment 1}

\begin{document}
	\lhead{\name}
	\chead{\thistitle}
	\rhead{\sid}

\section*{Problem 1}
	\begin{enumerate}[label=(\alph*)]
		\item 
		\begin{enumerate}[label=(\roman*)]
			\item 
			Procedure for $\textproc{Heavy}(i, j)$\footnotemark:
			\begin{algorithm}
				\caption{Heavy, returns the set of heavy elements within a range $i$, $j$, inclusive.}
				\label{heavy}
				\begin{algorithmic}[1]
					\Procedure{Heavy}{$i$, $j$}
						\If{$i = j$}
							\Return $\{A[i]\}$ \Comment{Base case of a single element, which is heavy on its own.}
						\EndIf

						\Statex
						\State $m \gets \floor*{\frac{i + j}{2}}$
						\Comment{Take the midpoint of $i$ and $j$.}

						\State $L \gets \textproc{Heavy}(i, m)$
						\Comment{Recurse on left half}
						
						\State $R \gets \textproc{Heavy}(m+1, j)$
						\Comment{Recurse on right half}

						\State Create Set $S$
						\label{heavy:newset}
						\Comment{Initialise an empty set, to be returned.}

						\ForAll{$e \in L \cup R$}
						\Comment{For each (distinct) element in the both $L$ and $R$}

							\State $c \gets $ count of $e$ in $A[i \dots j]$
							\label{heavy:count}
							\Comment{$O(j - i + 1)$, linear w.r.t. size of $A[i\dots j]$}

							\If{$c > \frac{3}{20} (j - i + 1)$}
							\label{heavy:cond}
								\State $S \gets S \cup \{e\}$ \label{heavy:add}
								\Comment{$e$ is (still) heavy, $O(2|S|)$}
							\EndIf
						\EndFor
						\Return $S$
						\Comment{Return the heavy elements for $A[i \dots j]$}
					\EndProcedure
				\end{algorithmic}
			\end{algorithm}

			\footnotetext{Algorithm construction assisted by \textit{redacted}.}

			\item
			The algorithm \textproc{Heavy} first checks the base case, input where $i = j$. For any base case, the set of heavy items is a singleton of the element $A[i]$ itself (since $1 \ge 1 = \ceil*{\frac{3}{20}}$).

			Next, we divide the problem into two subproblems $\textproc{Heavy}(i, m)$ and $\textproc{Heavy}(m+1, j)$, each of which return a set of heavy numbers from the respective subarrays $A[i \dots m]$ and $A[m+1 \dots j]$.

			We take the union of these sets ($L \cup R$) to ensure no duplicated elements, then walk through the set. For each distinct element, we count the occurrences in $A[i \dots j]$ (\algref{heavy}{heavy:count}) and check if the count $c$ satisfies the heaviness condition (\algref{heavy}{heavy:cond}). Note that the counting needs to make one pass through the array and is hence $O(j - i + 1)$. If the condition is satisfied, the element is added (\algref{heavy}{heavy:add}) to a separate set (\algref{heavy}{heavy:newset}).
		\end{enumerate}
		
		\item
		Let's do some induction!

		Let the inductive proposition be $I(n)$: for some $i,j$, such that $i < j$, $n = j - i + 1$, $\textproc{Heavy}(i, j)$ returns a set $S$ such that $e$ is heavy for all $e \in S$.

		\textbf{Base case} ($n = 1$). By the proposition, for $I(1)$, we have $i = j$ and we correctly return the set of heavy elements, which is the singleton set $\{A[i]\}$.

		\textbf{General case} ($n > 1$). Assume $I(k)$ is true for all $k < n$. Now we have $m = \floor*{\frac{i + j}{2}}$ and we split into subcases with sizes $m - i + 1 = \floor*{\frac{j - i}{2}} + 1$ and $j - (m + 1) + 1 = \ceil*{\frac{j - i}{2}}$.
		
		Since $\floor*{\frac{j - i}{2}} + 1 < n$ and $\ceil*{\frac{j - i}{2}} < n$, we know by assumption that $I$ is true and that $\textproc{Heavy}$ correctly returns the set of heavy elements of $A[i\dots m]$ and $A[m+1 \dots j]$ respectively.

		We then check each element $e$ of the union of the subresults and ensure that the element is still heavy based on the condition that $e$ occurs at least 15\% of the time.

		Note that non-heavy elements are correctly discarded. If some array element $e$ isn't heavy (occurs less than 15\%) on the left and isn't heavy on the right, then $e$ isn't heavy on the combined array.

		\item
		\textbf{Base Case} ($n = 1$). $T(1) = 1$. This is the cost of constructing $\{A[i]\}$.

		\textbf{General Case} ($n > 1$). $T(n) = T\parens*{\floor*{\frac{n-1}{2}} + 1} + T\parens*{\ceil*{\frac{n-1}{2}}} + c_1 n$. For the subcases, we are reusing the derivations $\floor*{\frac{n}{2}} + 1$ and $\ceil*{\frac{n}{2}}$ from part (b) by substituting $n - 1 = j - i$.

		Note that $|L|$, $|R|$, and thus $|L \cup R|$ are bounded by a constant. The maximum amount of heavy numbers for any $\textproc{Heavy}(i, j)$ for some $n = j - i + 1$ is 6. Why? We prove by contradiction. Suppose the set returned contains 7 or more distinct heavy elements. Each element would have to appear more than 15\% of the set, accounting for 105\% (of $n$). This is a contradiction since the set should only contain up to 100\% of items. This result allows us simplify the complexity of lines 9 to 14 to $c_1 n$ where the $n$ comes from line 10 and the $c_1$ comes from summing the constant operations (set union $L \cup R$, $S \cup \{e\}$, etc.).
		
		We then have $T(n) \le T\parens*{\floor*{\frac{n}{2}}} + T\parens*{\ceil*{\frac{n}{2}}} + c_1n \le 2T \le 2T\parens*{\ceil*{\frac{n}{2}}} + c_1n$. By the master theorem for inequalities, we have $c = \log_2 2 = 1$ and thus $T(n) = O(n \log n)$.
	\end{enumerate}
	 
\section*{Problem 2}
	\begin{enumerate}[label=(\alph*)]
		\item 
		\begin{enumerate}[label=(\roman*)]
			\item Let
			$
			X_i =
			\begin{cases}
				1 & \text{if $A[i]$ is a local minimum} \\
				0 & \text{otherwise}
			\end{cases}
			$.
			
			For end cases, we have $P(X_1 = 1) = P(X_n = 1) = \frac{1}{2}$. (Why?\footnotemark[2])
			For middle cases $X_i$, we consider $A[i-1]$, $A[i]$, $A[i+1]$. There are 3! permutations and exactly 2 of these permutations situate $A[i]$ as a local minimum. Thus we have $P(X_i = 1) = \frac{2}{3!} = \frac{1}{3}$.

			There are 2 end cases and $n - 2$ middle cases. The expected number of local minima in $A$ is hence $E\parens*{\sum_{i=1}^n X_i} = 2 \times \frac{1}{2} + (n - 2) \times \frac{1}{3} = \frac{n + 1}{3}$. (Source.\footnotemark[3])

			\footnotetext[2]{We can prove this by induction. Let $P_n$ denote the probability that the first element (left end) is a local minimum. Given an array $A$ with random permutation of $\angles{1, \dots, n}$ ($n \ge 2$), we have a base case of $P_2 = \frac{1}{2}$. Assume $P_{n-1} = \frac{1}{2}$. We formulate $P_n = \frac{1}{n} + \frac{n - 2}{n}P_{n-1}$, which by solving gives $\frac{1}{2}$.
			
			$P_n$ can be pictured as inserting a new larger element $n$ into the random permutation of $\angles{1, \dots, n-1}$. We retain the local minimum by inserting $n$ into the second index (corresponding to $\frac{1}{n}$) or inserting in the third index or beyond (corresponding to the recursive $\frac{n - 2}{n}P_{n-1}$).}

			\footnotetext[3]{Solution inspired from: \url{https://math.stackexchange.com/q/680660/615376}}
			\addtocounter{footnote}{2}

			\item 
			Generalising from the array end case and middle case, let $S$ be the data structure under inspection, $n$ be the number of nodes of $S$, $i$ be some index $1 \le i \le n$, and $m$ be the number of neighbours of $S[i]$ (the $i^{\text{th}}$ node of $S$). Let $X_i$ be the indicator random variable, indicating if $S[i]$ is a local minimum. Then $P(X_i = 1) = \frac{m!}{(m+1)!} = \frac{1}{m+1}$ (we fix the $i^\text{th}$ node and take the permutations of the $m$ neighbours).

			The nodes and expected values are broken down as follows:
			Firstly, the root. There is only one; with $P(X_1 = 1) = \frac{1}{3}$. Secondly, the leaves, each with one neighbour. There are $\frac{n + 1}{2}$. For $n \ge 2$, we have $P(X_i = 1) = \frac{1}{2}$, where $S[i]$ are leaves.
			
			This leaves\footnote{haha} the nodes in the middle levels, each of which have three neighbours. Theare are $n - 1 - \frac{n + 1}{2} = \frac{n - 3}{2}$. For $n \ge 3$, we have $P(X_i = 1) = \frac{1}{4}$, where $S(i)$ are middle nodes.

			The total expected value of local minima in a tree is then
			$$
			E(\sum_{i=1}^n X_i) =
			\begin{cases}
				0 & \text{if $n = 0$} \\
				\frac{1}{3} & \text{if $n = 1$} \\
				\frac{1}{3} + \frac{n + 1}{2} \cdot \frac{1}{2}
				+ \frac{n - 3}{2} \cdot \frac{1}{4} = \frac{9n + 5}{24} & \text{if $n \ge 3$}
			\end{cases}
			$$

			(Note that the $n = 2$ case doesn't exist by assumption.)

			\item
			We treat $(i,j)$ elements as nodes and use similar methodologies. Let $n = m^2$ be the number of elements of the matrix.

			For each node, if $M(i, j)$ has
			\begin{itemize}
				\item 2 neighbours (corners): $$
				X_{(i, j)} =
				\begin{cases}
					0 & \text{if } m = 1 \\
					4 & \text{if } m \ge 2
				\end{cases}
				$$

				\item 3 neighbours (edges, non-corners): $$
				X_{(i, j)} =
				\begin{cases}
					0 & \text{if } m = 1 \\
					4m - 8 & \text{if } m \ge 2
				\end{cases}
				$$

				\item 4 neighbours: $$
				X_{(i, j)} =
				\begin{cases}
					0 & \text{if } m = 1 \\
					(m - 2)^2 & \text{if } m \ge 2
				\end{cases}
				$$
			\end{itemize}

			The expected value of local minima is then
			$$
			\begin{cases}
				0 & \text{if } m = 1 \\
				4 \cdot \frac{1}{3} + (4m - 8) \cdot \frac{1}{4}
				+ (m - 2)^2 \cdot \frac{1}{5} = m - \frac{2}{3} + \frac{1}{5}(m - 2)^2 & \text{if } m \ge 2
			\end{cases}
			$$


		\end{enumerate}

		\item
		\begin{enumerate}[label=(\roman*)]
			\item Similar to the array case, but without the end points. For each middle element, fix the middle element and permute the other two: two cases per $X_i$. Let $X_i$ be the indicator random variable indicating if $A[i]$ is a saddle point. $$E\parens*{\sum_{i=1}^n X_i} = \frac{n - 2}{3}.$$

			\item We only consider the elements in the middle level. So $P(X_i) = 0$ for $i$ where $T[i]$ is a root or leaf.
			
			From a.ii, we know there are $\frac{n - 3}{2}$. Each of these nodes have 4 neighbours, strictly ordered. For some $T[i]$, we consider its parent, $T[i]$, and its two children. Fixing the parent and $T[i]$, we deduce that there are two permutations (with $T[i]$ as a saddle point) out of $4!$. Thus, $P(X_i = 1) = \frac{1}{12}$, and $E\parens*{\sum_{i=0}^n X_i} = \frac{n - 2}{12}$.
			
			\item We split this problem into multiple cases. For corners, there is 1 out of 3! permutations satisfying the constraint. For non-corner edges, 2 out of 4! permutations. For inner cells, 4 out of 5! permutations. The expected value is then
			\begin{align*}
				E\parens*{\sum_{i=0}^n} &=
				\begin{cases}
					0 & \text{if } m = 1 \\
					4 \cdot \frac{1}{3!} + (4m - 8) \cdot \frac{2}{4!}
					+ (m - 2)^2 \cdot \frac{4}{5!} & \text{if } m \ge 2
				\end{cases} \\
				&= 
				\begin{cases}
					0 & \text{if } m = 1 \\
					\frac{2}{3} + \frac{m - 2}{3}
					+ \frac{(m - 2)^2}{30} & \text{if } m \ge 2
				\end{cases}
			\end{align*}

		\end{enumerate}
	\end{enumerate}

\section*{Problem 3}
	\begin{enumerate}[label=(\alph*)]
		\item 
		\begin{enumerate}[label=(\roman*)]
			\item 
			There is at least one minimum, and it is not at infinity, since $x_i$ are defined to be real numbers. Thus, there exists $z_1$ such that $f(x)$ is monotonically decreasing.

			\item 
			
		\end{enumerate}
	\end{enumerate}

\section*{Problem 4}
	\begin{enumerate}[label=(\alph*)]
		\item $T(n) = O(n^{5/2})$
		\item $T(n) = O(n^{\log_2{9}})$ % c = log_2 9, f(n) = n^3 log_2 n
		\item $T(n) = O(n^2 \log n)$ % c = 2, f(n) = n(n+1)/2
		\item $T(n) = O(n^{\log_4{3}})$ % c = log_4 3, f(n) = 1
		\item $T(n) = O(n^{\log_7{2}} \log n)$ % c = log_7 2, f(n) = 2^log_7(n)
		\item $T(n) = O(n \log n)$ % c = log_3 9 = 2, f(n) = log_3 n!
	\end{enumerate}

\end{document}